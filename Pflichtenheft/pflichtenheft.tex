\documentclass[10pt,a4paper]{article}
\usepackage[utf8]{inputenc}
\usepackage[german]{babel}
\usepackage{amsmath}
\usepackage{amsfonts}
\usepackage{amssymb}
\usepackage{graphicx}
\usepackage[left=2cm,right=2cm,top=2cm,bottom=2cm]{geometry}
\author{Felix Helsch Julian Buchhorn}
\title{Pflichtenheft}
\begin{document}

	\begin{titlepage}
	
		\begin{center}
		
			\begin{Huge}
				\vspace*{80px}
				Pflichtenheft \\
				\vspace*{25px}
				\today
				\vspace*{50px}
			\end{Huge}
			
			\begin{huge}
				Cross-Compiler von Python nach C++
				\vspace*{25px}
			\end{huge}
			
			\begin{large}
				Version: 0.1
				\vspace*{50px}
			\end{large}		
			
			\begin{large}
				Felix Helsch \\
				Julian Buchhorn
			\end{large}									
			
		\end{center}
		
	\end{titlepage}
	
	\tableofcontents
	
	\newpage
	
	\section{Zielbestimmung}
	
	Es soll ein Cross-Compiler von Python nach C++ entwickelt werden. 
	Zu Beginn wird ein einfaches Testprogramm entwickelt, erstmal nur mit 
	dem Ziel zu verstehen wie ein Compiler grundsätzlich funktioniert. 
	
	\subsection{Musskriterien}
	
	\begin{list}{$\rightarrow$}{}
		\item Übersetzen eines einfachen Python Programms nach C++
		\item Erweiterbarkeit
	\end{list}		
	
	\subsection{Wunschkriterien}
	
	\begin{list}{$\rightarrow$}{}
		\item Vereinfachung von Rechenausdrücken
		\item 
	\end{list}
	
	\subsection{Abgrenzungskriterien}
	
	\begin{list}{$\rightarrow$}{}
		\item 
		\item 	
	\end{list}
	
	\section{Produkteinsatz}
	
	\subsection{Anwendungsbereiche}
		
	\subsection{Zielgruppen}
	
	\section{Produktübersicht}
	
	\section{Produktfunktionen}	
	
	\section{Produktdaten}
	
	\subsubsection*{/D10/}
	
		Befehlssätze der Programmiersprachen
		
	\subsubsection*{/D20/}
		
		Zuordnungen der Befehle aus Python zu denen aus C++

	\section{Produktleistungen}
	
	\subsubsection*{/L10/}
	
	\section{Qualitätsanforderungen}
	
	\begin{tabular}{|c|c|c|c|c|}
		\hline 
		Produktqualität & Sehr gut & Gut & Normal & Nicht Relevant \\ 
		\hline 
		Funktionalität & & & & \\ 
		\hline 
		Zuverlässigkeit & & & & \\ 
		\hline 
		Benutzbarkeit & & & & \\ 
		\hline 
		Effizienz & & & & \\ 
		\hline 
		Änderbarkeit & & & & \\ 
		\hline 
		Übertragbarkeit & & & & \\ 
		\hline 
	\end{tabular} 
	
	\newpage 
	
	\section{Benutzungsoberfläche}
	
	\section{Technische Produktumgebung}
	
	\subsection{Software}
	
	Python in der Version ??? \\
	C++11
	
	\subsection{Hardware}
	
	Desktop-Computer
	
	\subsection{Orgware}
	
	keine
	
	\section{Spezielle Anforderungen an die Entwicklungsumgebung}
	
	\subsection{Software}
	
		\begin{itemize}
			\item Betriebssystemunabhängig
			\item IDE
			\item \LaTeX
		\end{itemize}
	
	\section{Ergänzungen}
	
	\subsubsection*{/E10/}	
	
\end{document}