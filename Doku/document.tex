
\documentclass[
11pt,
titlepage,
a4paper,
ngerman,
captions=tableheading,
toc=bibliography, 
toc=listof,
numbers=noenddot
] {scrartcl}

\usepackage{ifpdf}                     
\usepackage[T1]{fontenc}       
\usepackage[utf8]{inputenc}    
\usepackage{microtype}
\usepackage{amsmath, amssymb, amstext}
%\usepackage{palatino}            % andere Schriftart
\usepackage[ngerman]{babel}       % Deutsch !
\usepackage[babel, german=quotes]{csquotes}
\usepackage{color}                
\usepackage{graphicx}             % Bilder einbinden ermöglichen
\usepackage{url}                  % Besserew
\usepackage[textsize=small, textwidth=0.8in]{todonotes} % \todo noch zu tun
\usepackage{multirow}             % Zusammenfassen von Tabellenzellen 
\usepackage{upgreek}              % nicht-kursive Griechische Buchstaben
\usepackage{textcomp}             % \textperthousand  (promille)
\usepackage{rotating}             % rotierte Information
\usepackage[format=plain, indention=1cm, scriptsize,font=sf, labelfont=bf, nooneline]{caption} 
\usepackage{subfigure}%veraltet??
                                  % modif Bildunterschriften 
\renewcommand{\captionfont}{\small \sffamily \slshape} 
\renewcommand{\captionlabelfont}{\small \sffamily \slshape \bfseries   }

\usepackage[section]{placeins}
\usepackage{pdflscape}

%Einheiten
\usepackage[separate-uncertainty = true, locale=DE, alsoload=synchem]{siunitx}
\DeclareSIUnit{\ev}{\text{eV}}
\DeclareSIUnit{\u}{\text{u}}
\usepackage{units}

% Kontrolle der Seitenränder:
\usepackage[inner=3.5cm, outer=3cm, top=3cm, bottom=2cm, includehead, includefoot]{geometry}
% Besser nicht global als Standard setzen
%\clubpenalty=10000   % letzte Zeile einer Seite nicht erste Zeile eines Absatzes
%\widowpenalty=10000  % letzte Zeile eines Absatzes nicht erste Zeile einer Seite

% Eigene Kopf/Fußzeilen
\usepackage{fancyhdr}
\pagestyle{fancy}

% Standard:
\fancyhead[EL]{\thepage}
\fancyhead[ER]{\slshape \nouppercase{\leftmark}}
\fancyhead[OL]{\slshape \nouppercase{\leftmark}}
\fancyhead[OR]{\thepage}

% Fußzeile: Datum,  Seine X von Y, gut während des Schreibens:
\fancyfoot[EC, OC]{}%{\today{} $\ \ast\ast\ast\ $ Seite \thepage{}  von \pageref{LastPage}}

%neue Pakete
\usepackage{ae}
\usepackage{icomma}

\usepackage{hyperref}                            %macht anklickbare Links
\definecolor{LinkColor}{rgb}{0,0,0.75}           %dunkelblau
%\definecolor{LinkColor}{rgb}{0,0,0}             %schwarz

\hypersetup{
	pdftitle = {LaTeX-Vorlage},
	pdfsubject = {LaTeX-Vorlage},
	pdfkeywords = {LaTeX, Vorlage, Uni Wuppertal},
	pdfauthor = {Ich},
	bookmarksnumbered = true,
	bookmarksopen = false,
	colorlinks = false,
	hypertexnames = true,
	a4paper = true,
	colorlinks=true,%
	linkcolor=LinkColor,%
	citecolor=LinkColor,%
	filecolor=LinkColor,%
	menucolor=LinkColor,%
	urlcolor=LinkColor,
	breaklinks=true	
}

%%%%Codedarstellung
% Custom colors
\usepackage{color}
\definecolor{deepblue}{rgb}{0,0,0.5}
\definecolor{deepred}{rgb}{0.6,0,0}
\definecolor{deepgreen}{rgb}{0,0.5,0}
\definecolor{bgcolor}{rgb}{0.95,0.95,0.95}

\usepackage{listings}

% Python style for highlighting
\newcommand\codestyle[1]{\lstset{
language=#1,
%basicstyle=\ttm,
otherkeywords={self},             % Add keywords here
%keywordstyle=\ttb\color{deepblue},
emph={MyClass,__init__,rld},          % Custom highlighting
emphstyle=\ttfamily\color{deepblue},    % Custom highlighting style
%stringstyle=\color{deepgreen},
%frame=tb,                         % Any extra options here
showstringspaces=false,            %
%%
keywordstyle=\bfseries\ttfamily\color{deepred},
identifierstyle=\ttfamily\color{black},
commentstyle=\color[RGB]{120, 191, 142},
stringstyle=\ttfamily\color[rgb]{0.81,0.17,0},%[RGB]{deepgreen},%[rgb]{0.627,0.126,0.941},{164,196,0}
basicstyle= \ttfamily\footnotesize\color[RGB]{49,102,145},
numberstyle=\scriptsize\color{black},
numbers=left,
stepnumber=1,
numbersep=10pt,
tabsize=2,
breaklines=true,
%prebreak = \raisebox{0ex}[0ex][0ex]{\ensuremath{\hookleftarrow}},
breakatwhitespace=false,
aboveskip={1.5\baselineskip},
  columns=fixed,
  upquote=true,
  extendedchars=true,
 frame=single,
rulecolor=\color{black},
backgroundcolor=\color{bgcolor},
}}


% External codefiles
\newcommand\codefile[2]{{
\codestyle{#1}
\lstinputlisting{#2}}}

%%%%%

%%neue Kommandos
\newenvironment{changemargin}[2]{%
\begin{list}{}{%
\setlength{\topsep}{0pt}%
\setlength{\leftmargin}{#1}%
\setlength{\rightmargin}{#2}%
\setlength{\listparindent}{\parindent}%
\setlength{\itemindent}{\parindent}%
\setlength{\parsep}{\parskip}%
}%
\item[]}{\end{list}}

%Sachen nebeneinander packen + \parindent Fox
\newlength{\myParindent} %Zwischenspeicher für \parindent
\setlength{\myParindent}{\the\parindent}

\newcommand{\colw}{0.63\linewidth}% ändern mit \renewcommand
\newcommand{\colmarg}{-1.2cm}
\newcommand{\twocol}[2]{
\begin{changemargin}{\colmarg}{\colmarg + 0.5cm}
\begin{minipage}{\colw}
#1
\end{minipage}
\begin{minipage}{\linewidth - \colw}
#2
\end{minipage}
\end{changemargin}
}
\newcommand{\colcap}[1]{ %setzt irgendwie \parindent auf 0, kA was das noch alles verhaut.
\vspace{-1.6em}
\captionof{figure}{#1}
\vspace{1.4em}
\setlength{\parindent}{\the\myParindent} %daher alten Wert zurücksetzen
}
\newcommand{\resetcol}{
\renewcommand{\colmarg}{-1.2cm}
\renewcommand{\colw}{0.63\linewidth}
}

\newcommand{\ufrac}[3]{\SI{#1}{$\unit{\frac{#2}{#3}}$} }
\newcommand{\PM}{&\hspace{-0,3cm}$\pm$&\hspace{-0,3cm}}


\newcommand{\gq}[1]{\glqq #1\grqq ~} %German Quotes

%%%%%%%%%%%%%%%%%%%%%%%%%%%%%%%%%%%%%%%%%%%%%%%%%%%%%%%%%%%%%%%
\newcommand{\ttt}[1]{\texttt{#1}} %Code Schrift
\DeclareUrlCommand\ttt{\texttt{\urlstyle{same}}}


%\newenvironment{nrcode}{\lstlisting}}{\endlstlisting}
\lstnewenvironment{nrcode}{%\renewcommand\lstlistingname{xxx}%
\lstset{numbers=left, frame=single}}%
{}
%[numbers=left, frame=single]
\newenvironment{source}{\quote}{\endquote}

%%Inline-Code-Setup
\newcommand{\inlinecodestyle}[1]{
\codestyle{#1}
\lstset{numbers=none , frame=none, basicstyle= \ttfamily\small\color[RGB]{49,102,145}, aboveskip={0.5\baselineskip}}%, backgroundcolor=\color{white}}
}
%%%%%%%%%%%%%%%%%%%%%%%%%%%%%%%%%%%%%%%%%%%%%%%%%%%%%%%%%%%%%%%

% Stil der Zitate und der Bibliographie
\usepackage[style=numeric, backend=biber, bibencoding=utf8, bibwarn=true, sorting=none]{biblatex}
\addbibresource{literatur.bib}



\title{ Einfacher? Crosscompiler von Python nach C++ }
\subject{Programmierpraktikum}
\author{Felix Helsch \\ Julian Buchhorn}
\date{\today}

\publishers{\parbox[b][8cm]{\textwidth}{\centering \textbf{Betreuer}\\Dr. Holger Arndt}}

\usepackage[version=3]{mhchem}
\usepackage{rotating}
\usepackage{booktabs}

\newcommand{\cc}[1]{\multicolumn{1}{c}{#1}} %Cell-Center
\newcommand{\C}[2]{\multicolumn{#1}{|c|}{#2}} %multiCol-wall
\newcommand{\mc}[2]{\multicolumn{#1}{c}{#2}} %multicol

\inlinecodestyle{python}
\lstset{ escapeinside= {\#(*} {*)} }
\lstset{
 rangeprefix= \#\\*,
 includerangemarker= false,
 texcl= false, %??
}
%%%%%%%%%%%%%%%%%%%%%%%%%%%%%%%%%%%%%%%%%%%%%%%%%%%%%%%%%%%%%%%%%%%%%%
\begin{document}

\pagestyle{empty}
\maketitle
\tableofcontents
%\clearpage
\listoffigures
%\clearpage
%\listoftables
%\clearpage
\lstlistoflistings

\clearpage
\pagestyle{fancy}
%%%%%%%%%%%%%%%%%%%%%%%%%%%%%%%%%%%%%%%%%%%%%%%%%%%%%%%%%%%%%%%%%%%%%%

\section{Einleitung}

Im Rahmen eines Programmierpraktikums an der Bergischen Universität Wuppertal haben wir einen Compiler von Python zu C++ geschrieben. Dabei ging es in erster Linie darum, zu verstehen wie ein Compiler funktioniert und nicht darum einen vollständigen Compiler zu schreiben.

Als vorläufiges Ziel hatten wir uns gesetzt ein einfaches Testprogramm mit grundlegenden Syntaxelementen zu übersetzen (vgl. Lst. \ref{pyprog}).

\codefile[caption= Python Testprogramm für den Compiler, label=pyprog]{python}{program.py}

Das Ziel haben wir erreicht. Unser Compiler kann in seinem momentanen Zustand das Python Programm übersetzen und liefert das C++ Programm in Listing \ref{cppprog}. Dabei mussten wir, wie erwartet, einige Annahmen über das Programm machen, um die Übersetzung zu vereinfachen.

\codefile[caption= Vom Compiler erzeugtes C++ Programm, label=cppprog]{c++}{program.cpp}


In den folgenden Kapiteln werden wir erklären wie man den Compiler verwenden kann, welche Programme benötigt werden und welche Features der Compiler im Moment unterstützt. Zum Schluss werden wir noch kurz auf die Struktur des Codes eingehen, den wir geschrieben haben.

Für eine detailliertere Erklärung verweisen wir auf die Bachelorarbeit von Felix Helsch verwiesen..., in welcher die Erstellung des Compilers fortgeführt wird.


%%%%%%%%%%%%%%%%%%%%%%%%%%%%%%%%%%%%%%%%%%%%%%%%%%%%%%%%%%%%%%%%%%%%%%

\section{Verwendung}

Um ein Python Programm nach C++ zu compilieren, muss man nur das Main-Programm \ttt{py2cpp.py} aufrufen und dabei den Namen des zu übersetzenden Programms als Kommandozeilenparameter angeben. Das heißt, dass zum Beispiel unser Testprogramm \ttt{program.py} mit dem folgenden Konsolenbefehl übersetzt werden kann:
\begin{lstlisting}
python py2cpp.py program.py
\end{lstlisting}
Dafür muss man allerdings die benötigten Programme installiert haben (s. Kapitel \ref{progs}) und beachten, dass das zu übersetzende Programm nur die momentan unterstützten Features enthält (s. Kapitel \ref{features}).

%%%%%%%%%%%%%%%%%%%%%%%%%%%%%%%%%%%%%%%%%%%%%%%%%%%%%%%%%%%%%%%%%%%%%%

\section{Installation} \label{progs}

Zuerst muss man das Archiv mit den Compilerdateien herunterladen \cite{py2cpp} und an einer beliebigen Stelle auspacken.
Außerdem müssen Python und C++ installiert sein (siehe nächster Abschnitt).
Innerhalb dieses Verzeichnisses kann der Compiler dann wie beschrieben aufgerufen werden, wenn sich die zu übersetzende Datei ebenfalls im selben Verzeichnis befindet.
Sonst sollte man entsprechend den absoluten Pfad angeben.


\subsection{Benötigte Programme}

Folgende Programme werden zur Verwendung des Compilers in jedem Fall benötigt, jeweils in der angegebenen Version oder einer neueren:

\begin{itemize}
\item \ttt{Python} 2.7
\begin{itemize}
\item \ttt{pandas} 0.15.2
\end{itemize}

\item \ttt{C++} 11
\item \texttt{Python-Runtime für Antlr} 4.4.1
\end{itemize}
%%

Die \ttt{Python-Runtime} ist in dem Compiler Archiv schon enthalten, braucht also nicht extra installiert werden. Falls man sie doch runterladen möchte findet man sie in der entsprechenden Sektion der \ttt{Antlr} Website \cite{pyruntime}.

\ttt{pandas} ist ein Package für Python welches der Compiler ebenfalls benötigt. Da \ttt{pandas} wiederum von den Packages \ttt{numpy}, \ttt{python-dateutil} und \ttt{pytz} abhängt, müssen diese ebenfalls installiert werden. Falls man diese Installationen nicht alle manuell durchführen will, bietet es sich hier an einen Package Manager zu benutzten (für Python unter Windows z.B. \ttt{Anaconda}). Für eine detaillierte Installationanleitung verweisen wir auf die Website von Pandas  \cite{pandas}.

\subsection{Optionale Programme}

Die nachfolgenden Programme werden prinzipiell nur benötigt, wenn man die Grammatik \ttt{py.g4} neukompillieren möchte:
\begin{itemize}
\item \ttt{Antlr} 4.4

\begin{itemize}
\item \ttt{Java} 1.6
\end{itemize}
\end{itemize}

\ttt{Antlr} bietet allerdings auch noch zusätzliche Funktionen für die sich eine Installation lohnen kann. Zum Beispiel kann \ttt{Antlr} auch grafische Ansichten des Parsetrees erzeugen (s. Abbildung \ref{g:tree}), welche sehr nützlich sind zur Fehlersuche, oder einfach nur um die Struktur des Parsetrees besser zu verstehen. Außerdem kann die Grammatik dann nicht nur für \ttt{Python} kompilliert werden, sondern auch für die anderen Sprachen welche \ttt{Antlr}  unterstützt.


Da \ttt{Antlr} in \ttt{Java} geschrieben ist, benötigt es auch eine \ttt{Java} Installation. Insgesamt ist die Installation von \ttt{Antlr} etwas aufwändiger (insbesonder unter Windows), weswegen wir dazu noch eine ausführlichere Anleitung verfasst haben, welche sich in Abschnitt \ref{Antlr} des Anhangs befindet. Insbesondere für die Installation unter Linux wollen wie aber auch auf die Anleitung auf der \ttt{Antlr} Website verweisen \cite{antlr}.




%%%%%%%%%%%%%%%%%%%%%%%%%%%%%%%%%%%%%%%%%%%%%%%%%%%%%%%%%%%%%%%%%%%%%%

\section{Features} \label{features}

\subsection{Unterstützte Syntaxelemente}

\subsection{Fehlererkennung}



%%%%%%%%%%%%%%%%%%%%%%%%%%%%%%%%%%%%%%%%%%%%%%%%%%%%%%%%%%%%%%%%%%%%%%

\section{Codestruktur}

%%%%%%%%%%%%%%%%%%%%%%%%%%%%%%%%%%%%%%%%%%%%%%%%%%%%%%%%%%%%%%%%%%%%%%
\appendix
\newpage


%%%%%%%%%%%%%%%%%%%%%%%%%%%%%%%%%%%%%%%%%%%%%%%%%%%%%%%%%%%%%%%%%%%%%%

\newpage

\section{Installation und Verwendung von Antlr} \label{Antlr}

\ttt{Antlr} ist prinzipiell ein Parser-Generator mit ähnlicher Syntax wie \ttt{Yacc} oder \ttt{Bison}. Der hauptsächliche Unterschied liegt darin das im Gegensatz zu \ttt{Yacc} einen recursiv-descent Parser erzeugt. Daraus ergibt sich der Vorteil, dass \ttt{Antlr} bessere Fehlermeldungen erzeugen kann, aber ein Nachteil ist, dass keine indirekte Linksrekursion möglich ist.

Für eine Erklärung der Syntax verweisen wir auf die \ttt{Antlr} Homepage \cite{antlrH}, in diesem Kapitel werden nur die Installation beschrieben und die benötigten Kommandozeilenaufrufe für \ttt{Antlr} bzw. den erzeugten Parser.

\subsection{Installation}

Um den Compiler ausführen zu können, muss Python in der Version 2.7 installiert sein. Außerdem werden die Python-Bibliotheken Numpy und Pandas, sowie deren Abhängigkeiten, benötigt. Am einfachsten ist es, hierfür einen Package-Manager wie zum Beispiel Anaconda zu installieren. \\ \\

Wenn die enthaltene Python Grammatik geändert werden soll und diese somit neu compiliert werden muss, ist die Installation etwas aufwändiger. Im Folgenden werden die Installationsschritte für Windows und Linux gezeigt.

\begin{enumerate}

\item Installation des Java Development Kit (SDK/JDK) in der Version 1.6 oder höher

\item Download des "`Complete ANTLR 4.4 Java binaries jar"'-Paketes (oder neuer) von der \href{http://www.antlr.org/download.html}{Download-Seite des ANTLR-Projektes}

\item Kopieren der heruntergeladenen Java-Bibliothek in den "`jre/lib/ext"'-Ordner des JDK (zum Beispiel: $"`C:\textbackslash Program Files (x86)\textbackslash Java\textbackslash jdk1.8.0\_25 \textbackslash jre\textbackslash lib\textbackslash ext"'$ \\ bzw. unter Linux: $/usr/lib64/jdk1.8.0\_25/jre/lib/ext/$)

\item Hinzufügen der jar-Datei zum Klassenpfad

\end{enumerate}


\subsection{Rekompilieren und grafische Synataxtree-Ansicht}



\begin{center}
 \includegraphics[width=0.8\linewidth]{Bilder/program_func_parse_tree2.png}
 \captionof{figure}{ Beispiel Parsetree } \label{g:tree}
\end{center}

%%%%%%%%%%%%%%%%%%%%%%%%%%%%%%%%%%%%%%%%%%%%%%%%%%%%%%%%%%%%

\section{Programm... } \label{py}

\codefile{python}{test.py}






\clearpage
\nocite{*}
\printbibliography

\end{document}
%%%%%%%%%%%%%%%%%%%%%%%%%%%%%%%%%%%%%%%%%%%%%%%%%%%%%%%%%%%%%%%%%%%%%%%%

\subsubsection*{Code Schrift}
Text mit \ttt{Code} \ttt{Schrift} .

\subsubsection*{Inline Code}
\begin{lstlisting}[caption=CAP]
inline code test
\end{lstlisting}

\vspace{4ex}
\subsubsection*{Numerierter Inline Code}
\begin{nrcode}[caption=CCC, label=pynr, linerange={a-c, d-e}]
import math
#\*a
2
nr code test #(*\label{txtline}*)
4
5
#\*b
6
7
#\*c
8
9
#\*d
10
11
#\*e
12
\end{nrcode}

Referenz für Code \ref{pynr} Zeile \ref{txtline}.

Referenz für Code \ref{pycode} Zeile \ref{test_arith}.